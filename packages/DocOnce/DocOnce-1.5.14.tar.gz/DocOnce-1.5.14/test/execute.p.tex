%%
%% File automatically generated using DocOnce (https://github.com/doconce/doconce/):
%% doconce format latex execute.do.txt --examples_as_exercises --execute
%%
% #ifdef PTEX2TEX_EXPLANATION
%%
%% The file follows the ptex2tex extended LaTeX format, see
%% ptex2tex: https://code.google.com/p/ptex2tex/
%%
%% Run
%%      ptex2tex myfile
%% or
%%      doconce ptex2tex myfile
%%
%% to turn myfile.p.tex into an ordinary LaTeX file myfile.tex.
%% (The ptex2tex program: https://code.google.com/p/ptex2tex)
%% Many preprocess options can be added to ptex2tex or doconce ptex2tex
%%
%%      ptex2tex -DMINTED myfile
%%      doconce ptex2tex myfile envir=minted
%%
%% ptex2tex will typeset code environments according to a global or local
%% .ptex2tex.cfg configure file. doconce ptex2tex will typeset code
%% according to options on the command line (just type doconce ptex2tex to
%% see examples). If doconce ptex2tex has envir=minted, it enables the
%% minted style without needing -DMINTED.
% #endif

% #define PREAMBLE

% #ifdef PREAMBLE
%-------------------- begin preamble ----------------------

\documentclass[%
oneside,                 % oneside: electronic viewing, twoside: printing
final,                   % draft: marks overfull hboxes, figures with paths
chapterprefix=true,      % "Chapter" word at beginning of each chapter
open=right,              % start new chapters on odd-numbered pages
10pt]{book}

\listfiles               %  print all files needed to compile this document

\usepackage{relsize,epsfig,makeidx,color,setspace,amsmath,amsfonts,amssymb}
\usepackage[table]{xcolor}
\usepackage{bm,ltablex,microtype}

\usepackage{graphicx}

\usepackage{ptex2tex}
% #ifdef MINTED
\usepackage{minted}
\usemintedstyle{default}
% #endif
\usepackage{fancyvrb}

\usepackage[T1]{fontenc}
%\usepackage[latin1]{inputenc}
\usepackage{ucs}
\usepackage[utf8x]{inputenc}

\usepackage{lmodern}         % Latin Modern fonts derived from Computer Modern

% Hyperlinks in PDF:
\definecolor{linkcolor}{rgb}{0,0,0.4}
\usepackage{hyperref}
\hypersetup{
    breaklinks=true,
    colorlinks=true,
    linkcolor=linkcolor,
    urlcolor=linkcolor,
    citecolor=black,
    filecolor=black,
    %filecolor=blue,
    pdfmenubar=true,
    pdftoolbar=true,
    bookmarksdepth=3   % Uncomment (and tweak) for PDF bookmarks with more levels than the TOC
    }
%\hyperbaseurl{}   % hyperlinks are relative to this root

\setcounter{tocdepth}{3}  % levels in table of contents

% prevent orhpans and widows
\clubpenalty = 10000
\widowpenalty = 10000

% Redefine double page clear to make it a blank page without headers
% (from BYUTextbook)
\makeatletter
\def\cleardoublepage{\clearpage\if@twoside \ifodd\c@page\else
\hbox{}
\thispagestyle{empty}
\newpage
\if@twocolumn\hbox{}\newpage\fi\fi\fi}
\makeatother
% These commands fiddle with the space left for page numbers in the TOC
% (from BYUTextbook)
\makeatletter
%\renewcommand{\@pnumwidth}{2em}
%\renewcommand{\@tocrmarg}{2.85em}
\makeatother

% Make sure blank even-numbered pages before new chapters are
% totally blank with no header
\newcommand{\clearemptydoublepage}{\clearpage{\pagestyle{empty}\cleardoublepage}}
%\let\cleardoublepage\clearemptydoublepage % caused error in the toc

% --- end of standard preamble for documents ---


% insert custom LaTeX commands...

\raggedbottom
\makeindex
\usepackage[totoc]{idxlayout}   % for index in the toc
\usepackage[nottoc]{tocbibind}  % for references/bibliography in the toc

%-------------------- end preamble ----------------------

\begin{document}

% matching end for #ifdef PREAMBLE
% #endif

\newcommand{\exercisesection}[1]{\subsection*{#1}}

\renewcommand{\u}{\pmb{u}}
\newcommand{\f}{\bm{f}}
\newcommand{\xbm}{\bm{x}}
\newcommand{\normalvecbm}{\bm{n}}
\newcommand{\ubm}{\bm{u}}

\newcommand{\x}{\pmb{x}}
\newcommand{\normalvec}{\pmb{n}}
\newcommand{\Ddt}[1]{\frac{D#1}{dt}}
\newcommand{\halfi}{1/2}
\newcommand{\half}{\frac{1}{2}}
\newcommand{\report}{test report}

% ------------------- main content ----------------------

\chapter{Automatic execution of code blocks}

Convert this document to \texttt{ipynb}, \texttt{latex} or \texttt{html} with e.g.:



\bsh
doconce format ipynb execute.do.txt --execute

\esh


\section{Code blocks in different languages}
\subsection{Python}

Python code




\bpy
for i in [1,2,3]:
  print(i)

\epy

\bpy
1
2
3
\epy



\bpy
print(i)

\epy

\bpy
3
\epy

\subsection{Bash}
Bash code




\bsh
if [ 1 -eq 1 ] ; then echo 1; fi
var_bash=10

\esh

\bpy
1
\epy



\bsh
echo $var_bash

\esh

\bpy
10
\epy

\subsection{Julia}

Julia code




\bjl
var_julia = 11
print(var_julia)

\ejl

\bba
\eba






\bjl
for n = 2:4
  var_julia = var_julia + n
end
print(var_julia)

\ejl

\bba
\eba

\subsection{R }

R code







\br
x <- 1:3
print(x)
#pdf("plot.pdf")
plot(x)
#dev.off()

\er

\bba
\eba

\subsection{Other languages}

Then Cython (with -h option so it is hidden in html/sphinx):




\bcycod
cpdef f(double x):
    return x + 1

\ecycod


Java code




\bjava
for (int i = 0; i < 5; i++) {
  System.out.println(i);
}

\ejava


Javascript code


\bjs
for (var x in [0,1,2]) {console.log(x)}

\ejs


matlab code




\bdat
for i = 1:2:10
  disp(A(i))
end

\edat


html code


\bhtml
<a href='test'></a>

\ehtml


C code












\bc
#include <stdio.h>

int main() {
  int i;

  for (i = 1; i < 11; ++i)
  {
    printf("%d ", i);
  }
  return 0;
}

\ec


\section{Code block environments}

Hidden execution cells (\texttt{pyhid}, \texttt{pycod-e}) can be used to perform operations (e.g.~imports, variable initializations) without showing any cell.
The \texttt{pyhid} environment executes and hides the cell in formats other than .ipynb:

This is a normal python block using the \texttt{pycod} environment


\bpycod
print('pycod')

\epycod

\bpy
pycod
\epy







The \texttt{pycod-e} environment executes but hides the cell also in .ipynb files:







\texttt{pycod} is a normal cell that should execute automatically when using \texttt{--execute}. Note that this cells relies on code executed in a previous hidden cell:






\bpycod
print(sys.version)
b = 2
c = a + b
print("The result is {}".format(c))
c

\epycod

\bpy
3.9.0 (default, Nov 15 2020, 14:28:56) 
[GCC 7.3.0]
The result is 3
\epy
\bpy
3\epy

% !split
\section{Special environments}

The \texttt{*-t} environment (e.g. \texttt{pycod-t}) formats a cell to text, and can be used to print an example




\bpycod
# This is a for-loop example
for i in [0,10]:
  print(i)

\epycod


The \texttt{*out}  (e.g. \texttt{pycod-out}) environment can be used to write a cell output:



\bpycod
# This is a text cell using pycod-t
1/0

\epycod




\bpy
# This is a output cell using the `pycod-out` environment
1/0: You cannot divide by zero

\epy

\bpy
# This is a output cell using the `pycod-out` environment
1/0: You cannot divide by zero
\epy

The \texttt{-h} postfix can be used in the \texttt{html} format to show a Show/Hide button that toggles the code visibility.

The \texttt{pyscpro} environment creates an interactive cell using \href{{https://github.com/sagemath/sagecell/}}{Sage} in the \texttt{html} format

% !split
\section{Plotting}

This is a cell that should plot and output:






\bpycod
from pylab import *
x = linspace(0, 10, 100)
plot(x, x*x)
show()

\epycod

\begin{center}
   \includegraphics[width=0.8\textwidth]{.doconce_figure_cache/595e59a8-29df-4e2f-8cb0-6d322c366116.pdf}
   \caption{figure}{$y=x^2$}
   \label{python_plot}
\end{center}




We can now refer to this plot as Figure~\ref{python_plot}.

To improve quality when exporting to {\LaTeX}, the following code has automatically
been run to enable PDF export in notebooks.




\bpycod
from IPython.display import set_matplotlib_formats
set_matplotlib_formats('png', 'pdf')

\epycod


% !split
\section{Ignore output}

Predefined output can be omitted by passing \Verb!--ignore_output! to DocOnce.
This will remove all environments ending with \texttt{out}.




\bpycod
a = 2
print(a)

\epycod

\bpy
2
\epy



\bpy
2

\epy

\bpy
2
\epy

% !split
\section{Code with errors}

If code contains errors, it will still be run and the exception shown as part
of the output:




\bpycod
for a in range(10)
    print(a)

\epycod

\bpy
  File "/tmp/ipykernel_671/3806628970.py", line 1
    for a in range(10)
                      ^
SyntaxError: invalid syntax
\epy

% !split
\section{Opening files}

The working directory is the same as the .do.txt file.
You may want to use \texttt{os.chdir} to change the directory.




\bpycod
with open("../LICENSE") as f:
    print(f.read())

\epycod

\bpy
Copyright (c) 2007-2015, Hans Petter Langtangen <hpl@simula.no> and
Simula Resarch Laboratory.

All rights reserved.

Redistribution and use in source and binary forms, with or without
modification, are permitted provided that the following conditions are
met:

    * Redistributions of source code must retain the above copyright
      notice, this list of conditions and the following disclaimer.

    * Redistributions in binary form must reproduce the above copyright
      notice, this list of conditions and the following disclaimer in
      the documentation and/or other materials provided with the
      distribution.

    * Neither the name of Simula Research Laboratory nor the names of
      its contributors may be used to endorse or promote products
      derived from this software without specific prior written
      permission.

THIS SOFTWARE IS PROVIDED BY THE COPYRIGHT HOLDERS AND CONTRIBUTORS
"AS IS" AND ANY EXPRESS OR IMPLIED WARRANTIES, INCLUDING, BUT NOT
LIMITED TO, THE IMPLIED WARRANTIES OF MERCHANTABILITY AND FITNESS FOR
A PARTICULAR PURPOSE ARE DISCLAIMED. IN NO EVENT SHALL THE COPYRIGHT
OWNER OR CONTRIBUTORS BE LIABLE FOR ANY DIRECT, INDIRECT, INCIDENTAL,
SPECIAL, EXEMPLARY, OR CONSEQUENTIAL DAMAGES (INCLUDING, BUT NOT LIMITED
TO, PROCUREMENT OF SUBSTITUTE GOODS OR SERVICES; LOSS OF USE, DATA, OR
PROFITS; OR BUSINESS INTERRUPTION) HOWEVER CAUSED AND ON ANY THEORY OF
LIABILITY, WHETHER IN CONTRACT, STRICT LIABILITY, OR TORT (INCLUDING
NEGLIGENCE OR OTHERWISE) ARISING IN ANY WAY OUT OF THE USE OF THIS
SOFTWARE, EVEN IF ADVISED OF THE POSSIBILITY OF SUCH DAMAGE.

Remarks:

The figure and movie files in doc/manual/* were made by the Doconce
author and is released under the same conditions as Doconce.


\epy

% ------------------- end of main content ---------------

% #ifdef PREAMBLE
\end{document}
% #endif

